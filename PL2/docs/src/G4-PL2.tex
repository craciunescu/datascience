%%%%%%%%%%%%%%%%%%%%%%%%%%%%%%%%%%%%%%%%%%%%%%%%%%%%%%%%%%%%%%%%%%%%%%%%%%%%%%%%
\documentclass[a4paper]{article}

\usepackage[utf8]{inputenc}
\usepackage[english]{babel}



\usepackage{Sweave}
\begin{document}
\title{G4.PL-2}
\author{David Emanuel Craciunescu \and Laura Pérez Medeiro}


\maketitle

%%%%%%%%%%%%%%%%%%%%%%%%%%%%%%%%%%%%%%%%%%%%%%%%%%%%%%%%%%%%%%%%%%%%%%%%%%%%%%%%

\section*{Introduction}

The aim of this second laboratory practice is to teach how to realise data association analysis using
R tools.

As the previous practice, it will be divided in two main parts:

%%%%%%%%%%%%%%%%%%%%%%%%%%%%%%%%%%%%%%%%%%%%%%%%%%%%%%%%%%%%%%%%%%%%%%%%%%%%%%%%%

\section*{Data association analysis for six shopping basket}

	\subsection*{Loading packages}

At the begining of the session, it was explained how to personalise the work enviroment. The first step
to do that, consist on searching the path where is the file called RProfile. Inside this file it will be
added the "foreign" package to the packages' list load by default.

Then, the file must be saved after executing the following command:
\begin{Schunk}
\begin{Sinput}
> getOption("defaultPackages")
\end{Sinput}
\begin{Soutput}
[1] "datasets"  "utils"     "grDevices" "graphics"  "stats"     "methods"  
\end{Soutput}
\end{Schunk}
In the followeing step, it will be checked if arules' packages could be found in the standard library.
\begin{Schunk}
\begin{Sinput}
> library()
\end{Sinput}
\end{Schunk}

As it can not be found there, it must be load. There are two options for doing it:

- Using global help for searching arules' package, where it can be found a .zip with the correspondat library
and also a .pdf which explained in detail how to use the package correctly. After that, we execute:

\begin{Schunk}
\begin{Sinput}
> #install.packages('D:\UNIVERSIDAD\4º\FUNDAMENTOS DE LA CIENCIA DE DATOS\arules_1.5-0.zip', repos=NULL)
> #library(arules)
\end{Sinput}
\end{Schunk}

In order to check that the package has been install correctyl, we can use the command:
\begin{Schunk}
\begin{Sinput}
> search()
\end{Sinput}
\begin{Soutput}
[1] ".GlobalEnv"        "package:stats"     "package:graphics" 
[4] "package:grDevices" "package:utils"     "package:datasets" 
[7] "package:methods"   "Autoloads"         "package:base"     
\end{Soutput}
\end{Schunk}

- The package can be also download directly from R, using the following command to install stable CRAN version:
\begin{Schunk}
\begin{Sinput}
> install.packages("arules")
\end{Sinput}
\begin{Soutput}
package ‘arules’ successfully unpacked and MD5 sums checked

The downloaded binary packages are in
	C:\Users\lperez\AppData\Local\Temp\Rtmp8QQe8Q\downloaded_packages
\end{Soutput}
\end{Schunk}
and to intall a current development version:
\begin{Schunk}
\begin{Sinput}
> #library("devtools")
> #install_github("mhahsler/arules")
\end{Sinput}
\end{Schunk}

	\subsection*{Loading data}
The data used for the analysis consist on the following matrix, composed by six shopping baskets:
{Bread, water,  milk,   oranges}
{Bread, water,  coffee, milk}
{Bread, water,  milk}
{Bread, coffee, milk}
{Bread, water}
{Milk}


%%%%%%%%%%%%%%%%%%%%%%%%%%%%%%%%%%%%%%%%%%%%%%%%%%%%%%%%%%%%%%%%%%%%%%%%%%%%%%%%%

\section*{Data association analysis for eigth car sales}
In this second part, it will be used a txt file which contains information about eitgh car sales.
The same analysis as it has been done before should be applied with this new data, but ussing a support
equal or higer than 40% and confidence equal or higher than 90%.


First, the .txt field must be generate with the data:
\begin{Schunk}
\begin{Sinput}
> data <- paste(
+ 	"X C N B,",
+ 	"X C B S,",
+ 	"X C N,",
+ 	"X N B S,",
+ 	"X C B,",
+ 	"N,",
+ 	"X C B,",
+ 	"S A,",
+ 	sep="\n")
> cat(data)
\end{Sinput}
\begin{Soutput}
X C N B,
X C B S,
X C N,
X N B S,
X C B,
N,
X C B,
S A,
\end{Soutput}
\begin{Sinput}
> write(data,file = "cars.txt")
\end{Sinput}
\end{Schunk}

Then, we read the file in order to create the transactions
\begin{Schunk}
\begin{Sinput}
> transactions= read.transactions(file="cars.txt",rm.duplicates=FALSE, format="basket", sep=",");